\documentclass{article}
\usepackage{hyperref}
\hypersetup{colorlinks=true, urlcolor=blue, pdfborderstyle={/S/U/W 1}}
\usepackage{xcolor}
\usepackage{titlesec}

\pagenumbering{gobble}

\titlespacing*{\section}{-20pt}{1.5ex}{-2ex}
\titleformat{\section}{\large\bfseries}{\thesection}{0.5em}{}

\setlength{\voffset}{-20pt}
\setlength{\textwidth}{420pt}
\setlength{\oddsidemargin}{-30pt}
\setlength{\topmargin}{0pt}
\setlength{\headsep}{0pt}

\setlength{\marginparsep}{40pt}
\setlength{\marginparwidth}{77pt}

\setlength{\footskip}{-20pt}
\setlength{\textheight}{700pt}


\title{ALEKSEY KONONOV\\\normalsize{kononal@gmail.com | github.com/llesha}}
\date{ }

\begin{document}

\begin{center}
\huge{\textbf{\quad \quad \quad \quad \quad \quad ALEKSEY KONONOV}}

\normalsize{\hspace{4.5cm} kononal@gmail.com $\diamond$ \href{https://github.com/llesha}{{\textbf{\underline{github}}}}} $\diamond$ \href{https://stackoverflow.com/users/19933941/llesha}{{\textbf{\underline{stackoverflow}}}}
\end{center}

\section*{SUMMARY}
\makebox[530pt]{\rule{570pt}{0.4pt}}\\

\noindent I am a software engineering graduate with experience in teaching and software development.
I have worked as a backend Kotlin and Java developer, geometry and algebra teacher, olympiad mathematics assistant, and intern at various companies. I have worked on projects using mainly Kotlin, Java, Godot and JS.
\section*{EDUCATION}
\makebox[530pt]{\rule{570pt}{0.4pt}}\\

\noindent\textbf{Higher School of Economics, Faculty of computer science}
\marginpar{2019 -- 2023\\ Moscow}

\noindent \textit{Bachelor of software engineering}
\section*{WORK EXPERIENCE}
\makebox[530pt]{\rule{570pt}{0.4pt}}\\

\noindent\href{https://shkolkovo.net/}{{\textbf{\underline{Shkolkovo school}}}}
\marginpar{February 2020 --\\ May 2021}
\\\textit{Algebra, geometry teacher and olympiad mathematics assistant}
\begin{itemize}
 \item 2019-2020 — geometry teacher (7-8 grade), olympiad mathematics assistant (5-8 grade);
 \item 2020-2021 — algebra teacher (8 grade), olympiad mathematics assistant (5-8 grade).
\end{itemize}

\noindent\href{https://www.appintheair.mobi/}{{\textbf{\underline{App in the Air}}}}
\marginpar{July 2021 --\\ October 2021}
\\\noindent\textit{Backend Java developer}
\\\\
\noindent The main goal is improving the electronic air tickets parsing algorithm, which uses ML and heuristics
for parsing multiple formats. I changed parts of the algorithm to support more formats, fixed failed test
cases. As a result, 5 new ticket formats where supported
\\\\
\noindent\href{https://cs.hse.ru/en/dse/}{{\textbf{\underline{HSE, Software Engineering Department}}}}
\marginpar{September 2021 --\\ June 2022}
\\\textit{Assistant at Software Construction course}
\\\\
\noindent  This job required reviewing Java assignments of students. Assignments included topics such as web sockets,
databases (apache derby), UI (javafx), multithreading. Near 80 assignments were reviewed with detailed
comments.
\\\\
\noindent\href{https://yandex.com/}{{\textbf{\underline{Yandex}}}}
\marginpar{November 2022 --\\ January 2023}

\noindent \textit{Backend intern Java Spring, Kotlin, Liquibase}
\\\\
\noindent Improved development tools in two teams: Billing and Sorting Centers (SC). In the Billing team, created
IntelliJ IDEA plugin, increased test coverage by 10-15\%. In the SC team, decreased a release time by
10\%, fixed flaky tests. Automatized a business process which was previously done manually.
\\\\
\noindent\href{https://inforion.ru/}{{\textbf{\underline{Inforion}}}}
\marginpar{February 2023 --\\ August 2023}

\noindent \textit{Backend Kotlin engineer}
\\\\
\noindent Working on Kotlin emulator called Kopycat.
\\\\
\noindent\href{https://yandex.com/}{{\textbf{\underline{Yandex}}}}
\marginpar{August 2023 --\\ November 2023}

\noindent \textit{Backend Kotlin intern}
\\\\
\noindent Refactored the whole project to use grpc-kotlin with coroutines, enabling asynchronous processing of requests. Fixed flaky e2e tests, added support for following features: replies and buttons in telegram and whatsapp, stickers in Ya.Messenger. Added Ya.Messenger-Markdown format converter.
\\\\
\section*{PROJECTS}
\makebox[530pt]{\rule{570pt}{0.4pt}}\\
\noindent\href{https://45.156.25.18/}{{\textbf{\underline{Geomc}}}}
\marginpar{October 2022 --\\ \textbf{Present}}

\noindent\textit{Kotlin, geometry}

\noindent Geomc is the system for creating and solving geometry problems using specifically designed domain-
specific language. It is inspired by Codeforces and sites alike, the initial idea is to create a similar system
but for geometry tasks. \href{https://github.com/konichiva-geom/GeometryChecker/}{{\textbf{\underline{Github}}}}
\\\\
\noindent\href{https://llesha.github.io/CrateGram/}{{\textbf{\underline{Crategram}}}}
\marginpar{May 2023 --\\ \textbf{Present}}

\noindent\textit{Kotlin multiplatform, PEG parser}
\\\\
\noindent Crategram is a tool for improving grammar writing skills. It is done with tasks where user has to construct
a PEG grammar.
\\\\
\noindent\href{https://llesha.github.io/regina-ide/}{{\textbf{\underline{Regina Programming Language}}}}
\marginpar{February 2022 --\\ August 2022}

\noindent\textit{Kotlin multiplatform, parsers}
\\\\
\noindent Interpreter code compiles into JVM and Javascript. Main feature of the language are declarative classes: class properties can be defined by other, not yet initialized properties and instances. Practical value of the language is in creation of concise SVG image generators. Additionally, an online IDE was created.
\\\\
\noindent\href{https://github.com/llesha/MapGen-KorGE}{{\textbf{\underline{Map generator for Pocket Palm Heroes}}}}
\marginpar{October 2021 --\\ April 2022}

\noindent\textit{Kotlin, C++, procedural generation, HoMM3}
 \\\\
\noindent The map generator for HoMM-like game, written with KorGE game engine. Creates a 2d map with castles, mines, roads, resources, obstacles and guards. The generated map can be opened in a map editor. Additionally, the map generation can be viewed incrementally in a window.
\\\\
\noindent\href{https://llesha.itch.io/}{{\textbf{\underline{Games on itch.io}}}}
\marginpar{December 2020 --\\ \textbf{Present}}

\noindent\textit{Godot, C\#, JS}
\\\\
\noindent All of my smaller projects and games are collected on itch.io page. Most notable ones are
and \href{https://llesha.itch.io/2d-house-generator}{{\textbf{\underline{2D house generator}}}} and the game \href{https://llesha.itch.io/sliding-platformer}{{\textbf{\underline{Sliding platformer}}}}.

\end{document}